\thispagestyle{plain}
\chapter*{Abstract}
\addcontentsline{toc}{chapter}{Abstract}
The explosion of the 4.0 industrial revolution is impacting many aspects of life, both to the economy and the society. Core technologies which include artificial intelligence, machine learning, 3D virtualization technology, digital transfer,... and especially, IoT. Internet of Things (IoT) has contributed greatly to this revolution. \par
With IoT, the devices (Things) will be connected together to perform certain functions. Communications subsystem in IoT applications is very important. However, there are still many issues that need to be overcomed and improved. For example, building a communication subsystem that limit the energy consumption, enhance communication distance, and optimize communication system operation such as reducing delay time, noise resistance, interference, reducing error rate, increasing the efficiency of using transmission channels is a must. \par 
There are some proposed outstanding communication technologies for IoT such as Sigfox, LoRa, DASH7,... Each technology has its own advantages and suitable for a typical range of applications, especially LoRa (Long Range) - known as long-range technology, low energy consumption and resistance to noise, Multipath resistance, Doppler effect. \par
To build a protocol stack to support communication from process to process, integrate more multi-access control mechanisms, security mechanisms and data integrity using the LoRa physical layer than the Alliance organization. 500 members who are scientists and engineers have launched LoRaWAN standard. Latest version v1.1 (2017). \par 
The project \textbf{"Development of Communications Subsystem for IoT Application based on LoRaWAN protocol"} proposed to build a communication model using LoRa communication technology, implementing LoRaWAN standard to serve IoT applications, WSNs as environmental monitoring and monitoring system, ... The main tasks and proposed results include: (i) analysis, architectural design, defining the functions of each component of the system; (ii) propose, design, implementing, verification each element including: LoRa Nodes using LoRa Module SX1276, MCU STM32L4, LoRa Gateway, and options - install LoRaWAN Backend; (iii) finally, I set up scenarios to measure and evaluate the performance of the system. 

\null\vfill
\begin{flushleft}
\textbf{Keywords}: LoRa; LoRaWAN; LPWAN; IoT; Gateway
\end{flushleft}
 


