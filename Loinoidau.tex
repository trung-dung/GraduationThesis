\thispagestyle{plain}

\chapter*{Lời nói đầu}
\addcontentsline{toc}{chapter}{Lời nói đầu}

Sự bùng nổ của cuộc cách mạng công nghiệp 4.0 đã - đang tác động rất nhiều lĩnh vực trong đời sống – kinh tế - xã hội. Những công nghệ cốt lõi làm nổi bật cách mạng 4.0 bao gồm: trí tuệ nhân tạo – học máy, công nghệ ảo hoá 3D, chuyển số số,… đặc biệt là lĩnh vực IoT (Internet vạn vật) có đóng góp rất lớn vào cuộc cách mạng này. \par
	Internet vạn vật hiểu đơn giản là các thiết bị (Things) sẽ được kết nối với nhau nhằm thực hiện những chức năng nào đó. Phân hệ truyền thông trong các ứng dụng IoT có vai trò rất quan trọng. Tuy nhiên, hiện nay còn nhiều vấn đề cần phải khắc phục, cải tiến, xây dựng nhiều hơn trong phân hệ truyền thông như hạn chế mức năng lượng tiêu thụ dành cho truyền thông, tăng cường khoảng cách truyền thông, và làm tối ưu sự hoạt động của hệ thống truyền thông như giảm thời gian trễ, kháng nhiễu, can nhiễu, giảm tỷ lệ lỗi, tăng hiệu suất sử dụng kênh truyền. \par 
	Một số công nghệ truyền thông nổi bật được đề xuất cho IoT gần đây như Sigfox, LoRa, DASH7,… Mỗi công nghệ có một ưu điểm nổi trội riêng và phù hợp với nhiều loại ứng dụng, đặc biệt LoRa (Long Range) – được biết đến như công nghệ truyền thông có khoảng cách truyền thông lớn, mức tiêu thụ năng lượng thấp và có khả năng kháng nhiễu, kháng Multipath, hiệu ứng Dopller. Để xây dựng một chồng giao thức nhằm hỗ trợ truyền thông từ tiến trình đến tiến trình, tích hợp thêm các cơ chế điều khiển đa truy nhập, cơ chế bảo mật, toàn vẹn dữ liệu sử dụng tầng vật lý tổ chức Alliance gồm hơn 500 thành viên là các nhà khoa học, kỹ sư đã đưa ra chuẩn LoRaWAN. Phiên bản mới nhất v1.1 (năm 2017). \par
	Đồ án có tên đề tài \textbf{“Thiết kế, phát triển phân hệ truyền thông cho các ứng dụng IoT dựa trên giao thức LoRaWAN”} đề xuất xây dựng một mô hình truyền thông sử dụng công nghệ truyền thông LoRa, thực thi chuẩn LoRaWAN nhằm phục vụ các ứng dụng IoT, WSNs như hệ thống quan trắc giám sát môi trường,… Nhiệm vụ chính và kết quả đề ra bao gồm: (i) phân tích, thiết kế kiến trúc, định nghĩa chức năng từng thành phần của hệ thống; (ii) đề xuất, thiết kế, cài đặt, kiểm tra từng phần tử gồm: LoRa Nodes sử dụng LoRa Module SX1276, MCU STM32L4, LoRa Gateway và lựa chọn – cài đặt LoRaWAN Backend; (iii) cuối cùng, em đặt ra những kịch bản nhằm đo đạc, đánh giá hiệu năng của hệ thống. \par
	Em xin cam kết những kết quả đạt được trong quá trình làm đồ án này là kết quả thực hiện độc lập của cá nhân em. 
%\vfill	
%\vspace*{\fill}

\null\vfill
\begin{flushleft}
\textbf{Từ khoá}: LoRa; LoRaWAN; LPWAN; IoT; Gateway.
\end{flushleft}
      