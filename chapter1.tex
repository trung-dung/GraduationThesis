\chapter{Đặt vấn đề}
Lịch sử đã chứng minh, việc phát triển đất nước, phát triển nền kinh tế hoặc các vấn đề xã hội khác đều xuất phát từ việc phát triển khoa học - kỹ thuật. Ví dụ điển hình là nước Mỹ nửa cuối thế kỷ XIX đầu thế kỷ XX với sự các ngành công nghiệp dầu mỏ của John Davison Rockefeller, ngành công nghiệp thép của Andrew Canegie, công nghiệp điện của J.P Morgan hay nền công nghiệp ô-tô của Ford,… \par
	Trong thời điểm hiện tại, sự phát triển bùng nổ của cuộc cách mạng công nghiệp 4.0 đã, đang và sẽ tác động trực tiếp cũng như gián tiếp tới các ngành kinh tế - tài chính – ngân hàng hoặc các lĩnh vực văn hoá, xã hội giáo dục khác. Các công nghệ cốt lõi được đưa ra trong cuộc cách mạng này là số hoá, chuyển đổi số, các công nghệ như dữ liệu lớn (Big-data), các bài toán về trí tuệ nhân tạo – học máy (machine learning), deep learning, công nghệ ảo hoá, Internet vạn vật (Internet of Things),… \par 
Đặc biệt, trong lĩnh vực điện tử - truyền thông, mạng công suất thấp – diện rộng (Low Power Wide Area Networks (LPWANs)) đặc trưng với các kết nối yêu cầu băng thông thấp, khoảng cách lớn và tiết kiệm năng lượng. Nó được xem là những giải pháp kỹ thuật tốt cho các ứng dụng Internet vạn vật (IoT) hoặc các ứng dụng Machine-to-Machine (M2M) bao gồm thành phố thông mình (đo đạc thông minh, điều khiển đèn đường), hoặc nông nghiệp thông mình (tưới tiêu thông minh, dự báo thu hoạch sớm),… Các ứng dụng được giới hạn về vấn đề chi phí và năng lượng. LPWAN có thể sử dụng như một mạng kín (private network) hoặc như một hạ tầng, được đưa ra bởi một bên thứ 3, cho phép các nhà cung cấp dịch vụ cho các ứng dụng IoT có thể phát triển ứng dụng trên phạm vi lớn mà không cần đầu tư nhiều công nghệ. Các đặc điểm chính của LPWAN như truyền thông lên tới 10 km giữa các thiết bị và gateway, có thời gian sống từ 5 – 10 năm. Tuy nhiên có trade-off với tốc độ bit-rate nhỏ hơn 5Kbps. Nội dung chương này, em đưa ra những vấn đề về kỹ thuật hiện đại được triển khai trong các hệ thống ứng dụng thực tế, đồng thời cũng đưa ra những kiến giải, tổng hợp về tình hình triển khai thực tế một số ứng dụng ở Việt Nam. Nhằm đưa ra các cơ sở, lý do lựa chọn hướng thiết kế, phát triển của đề tài đồ án tốt nghiệp này.

\section{Tổng quan về các công nghệ truyền thông}
Sự phát triển của IoT đã được ứng dụng trong nhiều lĩnh vực thực tế như mô hình nhà thông mình, thành phố thông minh, hệ thống nông nghiệp thông mình, du lịch thông mình,… Ý tưởng cơ bản của IoT là tất cả thiết bị - đồ vật, nói chung là vạn vật sẽ được kết nối với nhau – thông qua một công cụ truyền thông công nghệ nào đó. Khi ấy, các “Things” sẽ giao tiếp được với nhau, trao đổi thông tin và điều khiển lẫn nhau. Ví dụ, tại một cánh đồng, người chủ sẽ lắp đặt rất nhiều trạm tưới tự động, các trạm này sẽ thu thập dữ liệu thông tin về môi trường tại khu vực của trạm đó. Dữ liệu được gửi về một trạm có năng lực tính toán tốt hơn, tại đây, trạm tập trung này thực hiện tính toán và ra quyết định điều khiển tưới tiêu tới các thiết bị kia,… Điều này sẽ làm giảm thiểu chi phí nhân công lao động, và sẽ tối ưu nhất cho nông nghiệp, điều kiện sinh trưởng của cây trồng, …  \par
	Các nhà khoa học – kỹ sư hàng đầu đã đưa mô hình tham chiếu cho một hệ thống IoT gồm 5 tầng như sau:
	\begin{itemize}
	\item	Tầng một, tầng này được gọi là tầng thiết bị (vạn vật - Things). Các thiết bị sẽ có thể là bất kỳ các thiết bị nào đó. Lấy ví dụ nó có thể là một nút quan trắc trong mạng cảm biển không dây được trang bị các loại cảm biến và các cơ cấu chấp hành. Các thiết bị này có chức năng thực hiện thu thập thông tin từ môi trường hoặc chính thông tin trạng thái hoạt động của nó để gửi đi và thực hiện yêu cầu mà bộ điều khiển gửi đến,
\item	Tầng 2 là tầng truyền thông, tầng này có nhiệm vụ trao đổi thông tin giữa các thiết bị với nhau hoặc giữa các thiết bị tới các ứng dụng. Tuỳ thuộc vào loại ứng dụng mà tầng này có những giải pháp kỹ thuật khác nhau,
\item	Tầng 3 là tầng xử lý, tại tầng này, các thông tin nhận được từ “Things” sẽ được tiến hành xử lý, chọn lọc, giá trị hoá dữ liệu,
\item	Tầng 4 là tầng “học”, ở tầng này, với một lượng dữ liệu lớn mà thu được các công nghệ về Big – Data được xử lý, hoặc các model về Machine Learning – AI, Deep-learning được phát triển cung cấp API cho tầng trên cùng,
\item	Tầng cuối, là tầng ứng dụng, với tầng này, dữ liệu, hoặc API sẽ được sử dụng cho các ứng dụng cụ thể đã được đặt ra.
	\end{itemize}
Theo như mô hình IoT đã đưa ra thì phân hệ truyền thông có một vai trò đặc biệt quan trọng trong sự thành công của một hệ thống IoT, tuy nhiên có một số vấn đề đặt ra với các giải pháp kỹ thuật ở tầng này như sau:
\begin{itemize}
\item	Bandwidth/datarate: Băng thông và tốc độ dữ liệu dùng để đánh giá số lượng dữ liệu truyền đi trên một đơn vị thời gian. Băng thông được hiểu là khoảng phổ (Hertz) mà hệ thống có thể dùng để truyền tín hiệu. Datarate phụ thuộc vào băng thông của kết nối Internet. Băng thông và data-rate tỷ lệ thuận với nhau. Trong LoRaWAN data-rate được lựa chọn trade-offs với khoảng cách truyền thông và thời gian truyền của bản tin. LoRaWAN tạo ra các kênh truyền "ảo" với các data-rate khác nhau và chống nhiễu nhờ công nghệ điều chế trải phổ,
\item	Batery Life (Thời gian sống): Vấn đề năng lượng cấp cho các thiết bị cuối trong mạng IoT là một vấn đề vẫn chưa có giải pháp tối ưu. Với LoRaWAN các thiết bị có mức tiêu thụ công suất thấp. Ngoài ra để tăng tối đa hiệu quả sử dụng pin của các thiết bị cuối, LoRaWAN server điều kiển đầu ra RF (RF frequency) và các tỷ số đầu vào thông qua một tỷ số data-rate đáp ứng (adaptive data-rate),
\item	Range (khoảng cách truyền thông): Các công nghệ mới có mục tiêu cung cấp khả năng truy cập tới Internet tới người dùng/thiết bị từ một mạng đô thị lớn. Để có được khoảng cách truyền thông lớn thì phải đánh đổi việc tăng công suất thu-phát và dẫn đến mức tiêu thụ năng lượng tăng, thời gian duy trì pin cho thiết bị giảm xuống. Công nghệ LoRa và giao thức LoRaWAN có mục tiêu truyền thông với cự ly lớn và năng lượng tiêu thụ thấp, theo hướng tiếp cận năng lượng xanh. Với giao thức này, có thể truyền thông với khoảng cách 2-5km trong phạm vi đô thị, và cho phép lên đến 45km với vùng ngoại thành, nông thôn,
\item	Latency (trễ truyền thông): Ngày nay, các phương tiện truyền thông, các dịch vụ đều yêu cầu mức trễ là tối thiểu. Trong việc xây dựng mạng di động thế hệ thứ 5 (5G), một điều cần cân nhắc lại là sử dụng các nguồn tài nguyên hữu hạn để phục vụ các loại lưu lượng dịch vụ khác nhau dưới các môi trường khác nhau như thế nào? Trade-offs giữa truyễn thông downlink và thời gian sử dụng pin có thể giải quyết thông qua QoS classes trong các thiết bị LoRaWAN [3],
\item	Throughput (thông lượng): LoRaWAN cung cấp một thông lượng lớn hơn công nghệ dựa vào giao thức ALOHA, với độ phức tạp ở lớp MAC ít hơn.

\end{itemize}
Một số công nghệ truyền thông hay được sử dụng hiện nay cho các ứng dụng IoT như:
\begin{itemize}
\item	Bluetooth/LE là một công nghệ truyền thông không dây, kết nối giữa các thiết bị với data-rates tối đa là 1 Mpbs trong khoảng cách ngắn, theo lý thuyết là 100m với mức tiêu thụ năng lượng thấp. Sau một số bản phát hành, Bluetooth hiện tại là 4.0 nó có data rate cao hơn (lên tới 24 Mbps) với mức tiêu thụ năng lượng thấp hơn và thường được sử dụng nhằm kết nối các cảm biến và các bộ cơ cấu chấp hành trong môi trường IoT,
\item	DASH7 là một chồng giao thức dựa trên mô hình tham chiếu OSI nới mà các cảm biến được kết nối với nhau thông qua các tần số 433MHz, 868 và 915 MHz. DASH7 có mục tiêu cung cấp truyền thông cho các thiết bị lên tới 2Km với trễ thấp, hỗ trợ di động và pin nhiều lớp, mã hoá AES và data – rate lên tới 167 Kbps. Ngoài ra, DASH7 cũng đã đưa ra mô hình chuẩn theo kiến trúc phân lớp và các giao thức từ tầng vật lý đến tầng ứng dụng,
\item	Sigfox: là một hệ thống hướng tổ ong cho phép các thiết bị kết nối tới một trạm gốc (base-station) với software-defined cognitive radios sử dụng điều chế BPSK (Binary Phase Shift Keying). Sigfox sử dụng băng tần 868 MHz, được chia phổ ra thành 400 kênh rộng 100 Hz. Vùng phủ lên tới 30-50km vùng nông thôn và 3 – 10 km trong đô thị. Một access point có thể quản lý một triệu thiết bị xung quanh và mỗi thiết bị có thể gửi 140 bản tin một ngày với tốc độ 100 bps. Downlink chỉ có thể đứng trước uplink với một thiết bị phải đợi để nghe sự phản ứng của base station để biết điểm yêu cầu thu thập dữ liệu. Tuy nhiên sigfox không thích hợp lắm với các ứng dụng command và control.
\end{itemize}

\section{Tổng quan về WSN}
Mạng cảm biến không dây (wireless sensor network -WSN) là một mạng gồm các thiết bị cảm biến (nút cảm biến) được triển khai phân tán với số lượng lớn trong một không gian địa lý rộng để thực hiện các nhiệm vụ thu thập dữ liệu môi trường xung quanh chúng. Nút cảm biến là những thiết bị nhỏ gọn, có khả năng tự vận hành và tự cấu hình hoạt động để cảm nhận, ghi đo, tính toán các tham số môi trường. Các nút cảm biến có khả năng trao đổi dữ liệu với các nút khác trong mạng hoặc trao đổi dữ liệu với nút trung tâm (sink) hoặc trạm gốc (base station) qua các liên kết không dây Định nghĩa về mạng cảm biến không dây.
Mạng cảm biến không dây ra đời đáp ứng cho nhu cầu thu thập thông tin về môi trường tại một tập hợp các điểm xác định trong một khoảng thời gian nhất định nhằm phát hiện xu hướng hoặc quy luật vận động của môi trường. Chức năng của các nút trong mạng WSNs được thiết kế tùy thuộc vào từng ứng dụng cụ thể. Một số chức năng chính như: xác định giá trị các thông số tại nơi được lắp đặt, ví dụ như nhiệt độ, áp suất, độ ẩm, cường độ ánh sáng; phát hiện sự tồn tại của các sự kiện cần quan tâm và ước lượng các thông số của sự kiện đó; nhận dạng, phân biệt các đối tượng và theo dấu các đối tượng.	
Mạng WSN điển hình bao gồm một lượng lớn các nút cảm biến có giá thành thấp, công suất thấp và đa chức năng được triển khai một cách ngẫu nhiên hoặc theo cấu trúc trên một vùng địa lý rộng. Các nút cảm biến có khả năng tự cấu hình và hoạt động độc lập hoặc phối hợp tạo thành các nhóm để thực hiện các nhiệm vụ nhất định. Các nút cảm biến thường hoạt động trong các môi trường khắc nghiệt, độc hại, môi trường không thân thiện hoặc những nơi con người khó có thể tiếp cận. Do vậy, so với các mạng không dây truyền thống, mạng cảm biến không dây có các đặc tính riêng biệt cũng như các giới hạn sau:
\begin{itemize}
\item	Nút cảm biến có tài nguyên hạn chế: Năng lực xử lý yếu, bộ nhớ hạn chế, tốc độ trao đổi dữ liệu thấp. Nút mạng chỉ được cung cấp một nguồn năng lượng giới hạn. Trong nhiều ứng dụng, việc bổ sung năng lượng là không thể thực hiện được,
\item	Quy mô lớn: Kích thước của mạng cảm biến khác nhau tùy vào ứng dụng, một số mạng có số lượng nút cảm biến có thể lớn gấp nhiều lần mạng ad-hoc truyền thống. Ngoài ra, nút cảm biến cũng có thể được triển khai với mật độ rất dày đặc, khả năng sảy ra lỗi cao. Việc duy trì cấu trúc mạng, đồng bộ và duy trì sự hoạt động hiệu quả của mạng là một vấn đề có nhiều thách thức,
\item	Mô hình truyền thông mới: Khác với mô hình mạng không dây truyền thống thường sử dụng truyền thông điểm-điểm, các nút cảm biến sử dụng phương pháp truyền thông quảng bá, điểm-đa điểm. Dữ liệu cảm biến sẽ được chuyển tiếp qua nhiều nút mạng trước khi tới đích. Các giao thức và thuật toán trong các mạng không dây truyền thống cần được thiết kế lại để phù hợp với cấu trúc và đặc trưng của mạng WSN,
\item	Tính đa dạng trong thiết kế và sử dụng: Các thiết bị cảm biến có khuynh hướng được thiết kế dành riêng cho từng ứng dụng cụ thể. Với phạm vi ứng dụng rộng rãi, mạng WSN có thể có nhiều loại thiết bị vật lý khác nhau. Như vậy, các loại thiết bị này cần một sự điều chỉnh phần mềm ở một mức độ nào đó để có được hiệu quả sử dụng phần cứng cao. Môi trường phát triển chung là cần thiết để cho phép các ứng dụng riêng có thể xây dựng trên một tập các thiết bị mà không cần giao diện phức tạp.

\end{itemize}

\section{Lý do chọn đề tài}
Dựa trên những đặc điểm yêu cầu về truyền thông trong các hệ thống ứng dụng IoT đã được nêu trên. Dựa trên những đặc điểm nổi trội, thích hợp của công nghệ truyền thông LoRa, em quyết định lựa chọn công nghệ truyền thông LoRa thực hiện trong đồ án này. Công nghệ LoRa được bắt đầu được sử dụng từ năm 2015, cho đến nay cũng không phải là một công nghệ quá mới. LoRa cũng được nhiều sinh viên Bách Khoa chọn làm đồ án tốt nghiệp như đồ án [1], với đồ án này, tác giả đã thiết kế, chế tạo module truyền thông LoRa nhằm tích hợp vào hệ thống quan trắc giám sát môi trường. Với đồ án [2], tác giả đã đưa ra một giao thức đa truy nhập cho LoRa cũng ứng dụng trong mạng cảm biến không dây giám sát môi trường. Đặc biệt, với công trình công bố năm 2018 của nhóm nghiên cứu TS. Nguyễn Hữu Phát [3], nhóm tác giả đã giải mã sáng chế, ứng dụng LoRa vào nuôi tôm nước lợ ở Việt Nam. Các đồ án và công trình ứng dụng em đã nêu ra, đều sử dụng LoRa tích hợp vào các ứng dụng quan trắc, WSN. Tuy nhiên, mới chỉ dừng lại ở việc sử dụng LoRa thuần, và đưa ra giao thức đa truy nhập mà hiệu quả truyền thông không cao. Trong khi đó, Alliance đã công bố chuẩn giao thức LoRaWAN cho LoRa từ năm 2015. Nên với các yếu tố lý do về kỹ thuật cũng như thực trạng nên em quyết định lựa chọn đề tài này để thực hiện đồ án tốt nghiệp.

\section{Mục tiêu và kết quả dự kiến}
Những mục tiêu và nhiệm vụ nghiên cứu, thiết kế đặt ra là:
\begin{itemize}
\item	Tìm hiểu tổng quan về LoRa và LoRaWAN,
\item	Phân tích, lựa chọn giải pháp kỹ thuật - công nghệ để xây dựng mô hình hệ thống truyền thông mạng LoRaWAN,
\item	Xây dựng, triển khai thực tế kiến trúc mô hình truyền thông tuân theo chuẩn LoRaWAN ứng dụng nhằm phục vụ truyền thông trong các hệ thống quan trắc, đo lường tự động, bao gồm:
	\begin{itemize}
	\item	Các nodes cảm biến (thiết kế, xây dựng chương trình firmware điều khiển triển khai chuẩn LoRaWAN 1.0.3),
    \item	Khảo sát, lựa chọn, cài đặt cấu hình LoRa Gateway,
    \item	Tìm hiểu mô hình hoạt động của LoRa Server, lựa chọn, cài đặt, sử dụng một triển khải của LoRa Server để nhận dữ liệu từ LoRa Nodes. 
	\end{itemize}
\item	Tiến hành đo đạc, kiểm thử sự hoạt động của hệ thống. So sánh hiệu năng các tham số truyền thông so với lý thuyết hoặc các kết quả nghiên cứu mô phỏng mạng LoRaWAN đã công bố,
\item	Đề xuất thiết kế một hoặc hai hệ thống ứng dụng có sử dụng nền tảng mạng LoRaWAN ở Việt Nam.

\end{itemize}
Với những mục tiêu – công việc nghiên cứu đưa ra, những sản phẩm dự kiến đạt sau khi hoàn thành đồ án bao gồm:
\begin{itemize}
\item	Đề xuất, thiết kế - lựa chọn giải pháp kỹ thuật, triển khai xây dựng mô hình kiến trúc ứng dụng mạng LoRaWAN bao gồm (i) LoRa Node; (ii) LoRaGateway; (iii) LoRa Server. Nhằm phục vụ các ứng dụng quan trắc, đo đạc, cảnh báo,…
\item	Báo cáo thiết kế, và các tài liệu liên quan,…
\end{itemize}



\section{Những công việc chính}
Những công việc chính em sẽ thực hiện trong quá trình làm đồ án:
\begin{itemize}
\item	Khảo sát tình hình nghiên cứu, triển khai mạng LoRaWAN và các công việc – nghiên cứu có liên quan,
\item	Tìm hiều, phân tích, tổng hợp các cơ sở lý thuyết liên quan về LoRa và LoRaWAN,
\item	Phân tích đưa ra các định nghĩa đặc tả kỹ thuật, yêu cầu kỹ thuật (viết specifications),
\item	Thiết kế mô hình, kiến trúc hệ thống, lựa chọn công nghệ - kỹ thuật: 
	\begin{itemize}
	\item	Thiết kế kiến trúc, chức năng, đặc tả chức năng kỹ thuật, phi chức năng của LoRa Node (mạch truyền thông): Bao gồm các khối chức năng, nguồn, GPS, GSM, module LoRa, phân tích lựa chọn MCU, đặc điểm MCU, Led Indicators,… 
    \item	Mô tả sự lựa chọn – cài đặt LoRa-Gateway, phân tích quy trình cài đặt, hoạt động,…
    \item	Thực hiện xây dựng LoRa Network Server, LoRa Application Server tuân theo chuẩn LoRaWAN. 

	\end{itemize}

\end{itemize}
Trong quá trình thực hiện đồ án, để đạt được những công việc như đã đề ra, em đã sử dụng các phương pháp nghiên cứu dưới đây: \par
Phương pháp phân tích và tổng hợp lý thuyết: Phân tích và tổng hợp lý thuyết, các nguyên lý cơ bản được đào tạo trong trường Đại Học Bách Khoa Hà Nội có liên quan đến đồ án như kỹ thuật điện tử, xây dựng mạch điện tử. Lựa chọn, đánh giá xây dựng các khối kiến trúc và thiết kế lựa chọn các khối mạch. Các mảng kiến thức về thông tin số, xử lý số tín hiệu để tìm hiểu, tổng hợp liên kết nội dụng lý thuyết tầng vật lý của kỹ thuật điều chế Long Range LoRa; Và các giao diện kết nối (interfaces), thủ tục (procedure), chồng giao thức (protocol stack) để phân tích đánh giá, và triển khai cài đặt giao thức chuẩn LoRaWAN. \par 
Phương pháp nghiên cứu thực nghiệm: Phương pháp nghiên cứu, triển khai thực nghiệm được sử dụng nhằm tiến hành xây dựng bản mẫu hệ thống, thiết kế chương trình điều kiển. Đo đạc các thông số đảm bảo chất lượng từng phần tử hệ thống và đo đạc các tham số quyết định hiệu năng của mạng truyền thông LPWAN nói chung cũng như mạng LoRaWAN nói riêng. 
Phương pháp tham khảo tài liệu: Một phương pháp quan trọng nữa là phương pháp tham khảo tài liệu, dẫn chứng tài liệu. Phương pháp này cung cấp cho em một cái nhìn tổng quan, hiểu về các khía cạnh và tình hình nghiên cứu của đề tài hiện tại trong nước và thế giới. Nhằm tìm hiểu, phân tích và đưa ra các vấn đề, khía cạnh của đề tài mà các báo cáo trong nước và ngoài nước chưa đạt được, sau đó đưa ra ý tưởng cải tiến hoặc sửa lỗi. Đồng thời, tham khảo tài liệu, cũng giúp em có được những dẫn chứng khoa học chính xác, chặt chẽ và đựng nên một đề tài có tính đúng đắn và lập luận vững chắc hơn.  \par
Phương pháp tham khảo ý kiến chuyên gia: Học hỏi các thầy cô, các chuyên gia, bạn bè có kiến thức, chuyên môn trong lĩnh vực điện tử – truyền thông, đặc biệt là thiết kế hệ thống mạng cảm biến, mạng truyền thông công suất thấp – khoảng cách lớn. Nhằm cung cấp nhiều thông tin, kinh nghiệm hữu ích giúp em có nhiều kiến thức và tránh phải những lỗi sai cơ bản.  

\section{Cấu trúc của báo cáo}
Cấu trúc của báo cáo đồ án này bao gồm 7 chương: (i) chương đặt vấn đề nêu lên tổng quan về các công nghệ truyền thông ứng dụng trong IoT hiện nay. Tình hình việc ứng dụng chuẩn LoRaWAN Alliance ở Việt Nam từ đó đưa ra lý do chọn đề tài, mục tiêu đề tài và các kết quả dự kiến; (ii) Cơ sở lý thuyết, chương này đưa ra những kiến thức lý thuyết cơ bản nhất về công nghệ điều chế LoRa cũng như những đặc điểm của chuẩn LoRaWAN Alliance; (iii) Phân tích, mô tả kiến trúc hệ thống – trong chương này, em đưa ra mô hình kiến trúc phân hệ truyền thông ứng dụng trong các hệ thống IoT, WSN phục vụ quan trắc mô trường, nông nghiệp thông minh, thuê xe thông minh,… dựa trên chuẩn giao thức LoRaWAN; (iv) phân tích, thiết kế mạch truyền thông LoRa Node – chương này em đưa ra những phân tích yêu cầu chức năng, giải pháp kỹ thuật và thiết kế của một mạch truyền thông đóng vai trò là node trong hệ thống truyền thông; (v) phân tích, thiết kế LoRa gateway – đưa ra kết quả sự khảo sát về các loại gateway hiện có, phân tích, so sánh lựa chọn gateway và đưa ra quá trình cài đặt, sử dụng và bảo trì gateway lora; (vi) phân tích, thiết kế khối LoRa Backend – đề xuất giải pháp cho LoRa Backend; (vii) Triển khai, kiểm thử và kết quả - chương này đưa ra những báo cáo triển khai thực tế của em, các kết quả về kiểm thử và hoạt động của hệ thống; Cuối cùng là phần kết luận chung, danh sách tài liệu tham khảo và các phụ lục quan trọng. 
\section{Kết luận}
Trong chương này, em đã đưa ra kết quả báo cáo quá trình tìm hiểu, khảo sát các công nghệ truyền thông dùng trong IoT, WSN. Khảo sát về thực trạng triển khai mạng LoRaWAN ở Việt Nam trong nghiên cứu cũng như ứng dụng. Từ đó em đã nêu lên lý do chọn đề tài, các mục tiêu, công việc cần triển khai, và các kết quả dự kiến. Sang chương tiếp, em sẽ trình bày về các cơ sở lý thuyết quan trọng hỗ trợ trực tiếp trong quá trình thực hiện làm đồ án bao gồm lý thuyết về kỹ thuật điều chế trải phổ chirp, điều chế LoRa và các đặc điểm chính của chuẩn LoRaWAN Alliance. 